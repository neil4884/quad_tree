% Preamble
\documentclass[10.5pt, thai, a4paper, notitlepage]{article}

% Packages
\usepackage[top=2cm,left=2.5cm,right=2.5cm,bottom=3cm]{geometry}
\usepackage{flushend}
\usepackage[none]{hyphenat}
\usepackage{titlesec}
\usepackage{multicol}
\usepackage{lineno}
\usepackage{color}
\usepackage{fontspec}
\usepackage{polyglossia}
\usepackage{amsfonts, amsmath, ragged2e}
\usepackage{lipsum}
\usepackage{tikz}

\setdefaultlanguage{english}
\setotherlanguages{thai}
\setmainfont[script=thai]{Laksaman}
\setmonofont{Consolas}
\XeTeXlinebreaklocale "th"

\renewcommand\thesection{\arabic{section}.}
\renewcommand\thesubsection{\thesection\arabic{subsection}.}

\justifying
\titleformat{\section}{\normalfont \bfseries \raggedright}{\thesection}{1em}{}
\titleformat{\subsection}{\normalfont \itshape \raggedright}{\thesubsection}{1em}{}

\title{\fontsize{14pt}{14pt} \textbf{โครงสร้างข้อมูล Quadtree และการประยุกต์ใช้งาน}}
\author{
    \begin{tabular}{r l}
        \small ธีรินท์ เพ็ชร์รัตน์   & \small 6430185021 \\
        \small พิพรรธ จงพิพัฒน์ชัย   & \small 6431333721 \\
        \small วิวรรษธร ฐิตสิริวิทย์ & \small 6432158421
    \end{tabular}
}

\date{}

% Document
\begin{document}

    \maketitle


    \section{บทนำ} \label{sec:introduction}


    \section{โครงสร้างข้อมูล Space Partitioning Tree} \label{sec:spacepart}


    \section{รูปแบบการเก็บข้อมูลของ Quadtree} \label{sec:data}

    \begin{center}
        \begin{tikzpicture}
    \node[circle, draw] {$Parent$}
    child {node[circle, draw] {Node}
    child {node[circle, draw] {SW}}
    child {node[circle, draw] {NW}}
    child {node[circle, draw] {NE}}
    child {node[circle, draw] {SE}}
    }
\end{tikzpicture}
    \end{center}


    \section{การประยุกต์ใช้งาน Quadtree} \label{sec:app}

    \subsection{การตรวจสอบการชนบนระนาบ} \label{subsec:collision}


    \clearpage


    \section{บรรณานุกรม} \label{sec:bib}

\end{document}